% Robert F.K. Martin
% ID 1505151
% CSci 5031
% Homework Set N

\documentclass[11pt]{article}
%\RequirePackage{vmargin}
%\setpapersize{USletter}
%\setmarginsrb{1in}{1.5in}{1in}{0in}{\headheight}{\headsep}{\footheight}{\footskip}
\usepackage{setspace}
\usepackage{homework}
\usepackage{macros}

% detect interpreter: pdflatex or latex
\newif\ifpdf
\ifx\pdfoutput\undefined
  \pdffalse
\else
  \pdfoutput=1
  \pdftrue
  \pdfcompresslevel=9
\fi

% user: add packages you need
\usepackage{amsmath}

% user: set counter depth for lists
%\setcounter{page}{1}

% user: set style of equation numbering
\numberwithin{equation}{section}  % (sec.number)
\renewcommand{\thetable}{\arabic{table}}
\renewcommand{\thefigure}{\arabic{figure}}

\singlespacing

\begin{document}

% pick type of your preferred page style. UMN requires page numbers.
%\pagestyle{headings}


\noindent Robert F.K. Martin\\
ID 1505151\\
Csci 5302\\
06 Mar 2006\\
Homework 2
\hline
\vspace*{0.25in}
\noindent\textbf{Problem 1a:\\
$A=\left( 3 -1 0;-6 1 -1 0 -3 -4\right)$ and $b=\left(1;1;-1\right)$.\\
Give the infinity norm of $A$, $||A||_{\inf}$, and give a vector \bf{x} such that $||x||_{\inf}$=1 and $||Ax||_{\inf}=||A||_{\inf}$.}
\\
\noindent The infinity norm is the largest row sum using absolute values. That gives 8 for the middle row.
\bf{x}=$\left(-1 1 -1\right)$.
\vspace*{0.25in}
\noindent\textbf{Problem 1b: }
\vspace*{0.25in}

\noindent\textbf{Problem 1c: }
\vspace*{0.25in}

\noindent\textbf{Problem 1d: }
\vspace*{0.25in}

\noindent\textbf{Problem 1e: }
\vspace*{0.25in}

\noindent\textbf{Problem 3a: Write out Newtown's iteration for solving the following equations:\\
a) $x^3-2x-5=0}
\\
\noindent $x_1=x_0+\frac{x_0^3-2x_0-5}{3x_0^2-2}$

\vspace*{0.25in}

\noindent\textbf{b) $e^{-x}-x=0$}
\\
\noindent $x_1=x_0+\frac{e^{-x}-x}{something}$

\vspace*{0.25in}

\noindent\textbf{c) $x \sin(x)-1=0$}
\\
\noindent $x_1=x_0+\frac{x \sin(x)-1}{\sin{x}-x \cos(x)}$
\vspace*{0.25in}

\noindent\textbf{Problem 2d: }
\vspace*{0.25in}

\noindent\textbf{Problem 1e: }
\vspace*{0.25in}

\noindent\textbf{Problem 3: }

\vspace*{0.25in}

\noindent\textbf{Problem 4: }

\end{document}










