% Robert F.K. Martin
% ID 1505151
% CSci 5031
% Homework Set N

\documentclass[11pt]{article}
%\RequirePackage{vmargin}
%\setpapersize{USletter}
%\setmarginsrb{1in}{1.5in}{1in}{0in}{\headheight}{\headsep}{\footheight}{\footskip}
\usepackage{setspace}
\usepackage{homework}
\usepackage{macros}

% detect interpreter: pdflatex or latex
\newif\ifpdf
\ifx\pdfoutput\undefined
  \pdffalse
\else
  \pdfoutput=1
  \pdftrue
  \pdfcompresslevel=9
\fi

% user: add packages you need
\usepackage{amsmath}

% user: set counter depth for lists
%\setcounter{page}{1}

% user: set style of equation numbering
\numberwithin{equation}{section}  % (sec.number)
\renewcommand{\thetable}{\arabic{table}}
\renewcommand{\thefigure}{\arabic{figure}}

\singlespacing

\begin{document}

% pick type of your preferred page style. UMN requires page numbers.
%\pagestyle{headings}


\noindent Robert F.K. Martin\\
ID 1505151\\
Csci 5302\\
08 Feb 2006\\
Homework 1
\hline
\vspace*{0.25in}
\noindent\textbf{Problem 1a: Compute $2.04 \times 10^{+1}$ - $4.5 \times 10^{-1}$.}

\noindent 20.40-0.45=19.95. Using the round-to-even rule, this becomes 2.00 $\times 10^{+1}$. Or, if we shift the second number to have the same mantissa as the first, we lose precision and end up with $2.04 \times 10^{+1} - 0.04 \time 10^{+1} = 2.00 \time 10^{+1}$ which is the same result.
\vspace*{0.25in}

\noindent\textbf{Problem 1b: Compute $9.04 \times 10^{+9}$ - $9.55 \times 10^{+8}$.}

\noindent 9,040,000,000-955,000,000=8,085,000,000. Again, using the round-to-even rule, this becomes 8.08 $\times 10^{+9}$. Or, if we shift the second number to have the same mantissa as the first, we lose precision and end up with $9.04 \times 10^{+9} - 0.95 \times 10^{+9} = 8.09 \times 10^{+9}$. This method gives a different solution.
\vspace*{0.25in}

\noindent\textbf{Problem 1c: Compute $1.44 \times 10^{-8}$ - $1.40 \times 10^{-8}$.}

\noindent 0.0000000144-0.0000000140=0.0000000004. This would be 0.40 $\times 10^{-9}$, but since we don't allow sub-normal numbers, this is an underflow and is zero.
\vspace*{0.25in}

\noindent\textbf{Problem 1d: Compute 1c allowing for sub-normal numbers.}

\noindent Allowing for sub-normal numbers, this would be as above, 0.40 $\times 10^{-9}$.
\vspace*{0.25in}

\noindent\textbf{Problem 1e: Give an example of a positive floating pt number s.t $fl(a \ast a) > 0$, but $fl(a \ast a \ast a) = 0$.}

\noindent Let $a=1.00 \times 10^{-4}$. Then, $a^2=1.00 \times 10^{-8}$, but $a^3=1.00 \times 10^{-12}$, which is an underflow and becomes zero.
\vspace*{0.25in}

\noindent\textbf{Problem 1f: How many distinct positive numbers can be represented?}

\noindent There are 9 different choices for $X$, 10 each for $Y$ and $Z$, and 19 for $W$(That's 9 for positive exponents, 9 for negative exponents, and one for 0). Multiplying them all we have 17100. If we consider that we could represent zero, then we would have 17101.
\vspace*{0.25in}

\noindent\textbf{Problem 1g: How many additional positive numbers can be represented allowing for sub-normal numbers?}

\noindent Allowing $X$ to be 0 only changes the smallest numbers. The smallest number previously was 1.00 $\times 10^{-9}$. So allowing for sub-normal numbers, the next smallest would be 0.99 $\times 10^{-9}$. We continue getting smaller until we reach 0.00 $\times 10^{-9}$. This means there are 100 additional numbers in this system.
\vspace*{0.25in}

\noindent\textbf{Problem 1h: What is the unit roundoff for this system? Does it change for sub-normal numbers}

\noindent We want to find an $\epsilon_{mach}$ s.t. $1+\epsilon_{mach}=1.01$. The smallest such $\epsilon_{mach}=5.00 \times 10^{-3}$. This does not change when allowing for sub-normal numbers.
\vspace*{0.25in}

\noindent\textbf{Problem 1g: What is the underflow limit? Does it change for sub-normal numbers?}

\noindent The smallest number is $1.00 \times 10^{-9}$. When allowing for sub-normal numbers, the limit is $0.01 \times 10^{-9}$.
\vspace*{0.25in}

\noindent\textbf{Problem 2a: Compute the expression $(x+h)^2 - x^2$ using the previous number system with $x=2.22 \times 10^{2}$ and $h=1.00 \time 10^{0}$.}

\noindent $(x+h)^2 = 59729 = 5.97 \times 10^{4}$.\\
$x^2 = 59284 = 5.93 \times 10^{4}$.\\
$(x+h)^2 - x^2 = 0.04 \times 10^{4} = 4.00 \times 10^2$.

\vspace*{0.25in}

\noindent\textbf{Problem 2b: Compute the expression $[(x+h) - x] \cdot [(x+h) + x]$ using the previous number system with $x=2.22 \times 10^{2}$ and $h=1.00 \time 10^{0}$.}

\noindent This reduces to $h \cdot [2x + h]$ which is $1.00 \times 10^0 \cdot [4.44 \times 10^2 + 1.00 \times 10^0] = 4.45 \times 10^2$.

\vspace*{0.25in}

\noindent\textbf{Problem 2c: For each of the previous, compute the absolute error, the relative error, and the number of digits of accuracy.}\\
\noindent For 2a, the absolute error is 400-445=-45. The relative error is $\frac{-45}{445}=-0.10$.\\
For 2b, the absolute error is 445-445=0. Thus, the relative error is also 0.

\vspace*{0.25in}

\noindent\textbf{Problem 3a: Compute the value of $s_3(x)$ for $x=\frac{1}{4},-1.000001199$, and give the absolute and relative errors for each.}

\noindent $s_3(0.25) = 1.283854197$, $exp(0.25)=1.28402514$. The absolute difference is approximately -0.0001713 and relative difference is approximately -0.0001333.\\
$s_3(-1.000001199) = 0.367879$, $exp(-1.000001199)=0.333333$. The absolute difference is approximately -0.0345462 and relative difference is approximately -0.0093907.
\vspace*{0.25in}

\noindent\textbf{Problem 3b: For what range of values would you expect the approximation to yield an absolute error of less than $10^{-2}$. What about the relative error?}

\noindent If we assume that the major contributor to the error is the first term not used, then we can approximate the range cutoff to be $\frac{x^4}{n!} < |10^{-2}|$. Solving for $x$, we get $\pm$0.7. The relative error will be shifted to the right as $exp(x)$ gets smaller as $x$ goes to zero.  
\vspace*{0.25in}

\noindent\textbf{Problem 4a: Let $p(x) = x^3 - 2.5x^2 + 2x-0.5011002$. When using a flea market calculator to find the root of $p$, you get $\tilde{x}=1.0$. What is the backward error?}

\noindent Backward error is $\tilde{x} - x$, with $x$ being a true root of $p$. There are 3 roots of $p$, which were solved by Mathematica. They are 0.504481, 0.950587, 1.044933. The minimum backward error is $1.0 - 1.044933=-0.044933$. The maximum backward error is $1.0-0.504481=.0495519$.
\vspace*{0.25in}

\noindent\textbf{Problem 4b: Which interval would you use to start with for bisection? [-1,0], [0,1], [1,2]? Give an explanation.}

\noindent I would use [1,2]. The requirement for bisection is that there is a sign change in the interval, or that the sign of the two ends of the intervals are not the same. $p(-1)=-6.0011002$. $p(0)=-0.50011002$. $p(1)=-0.0011002$. $p(2)=1.4988998$. The only sign change is in the interval I selected.
\vspace*{0.25in}

\noindent\textbf{Problem 4c: Solve $p(x)=0$ using Newton's method starting at $x_0=1$. If it fails, explain why.}

\noindent $p'(x)=3x^2-5x+2$. Evaluted at $x_0=1$, $p(x_0)=0$. Since Newton's formula is $x_1=x_0+\frac{p'(x_0)}{p(x_0)}$, all iterations of Newton's method will be the same, i.e. $x_0=x_1=x_2 \ldots x_n=1$.
\vspace*{0.25in}

\noindent\textbf{Problem 4d: Let $x=0.95$ and let $\tilde{p}(x)=x^3-2.5x^2+2x-0.5$. Evaluate $p(x), p(\tilde{x}), \tilde{p}(x), and \tilde{p}(\tilde{x})$. Using these numbers, give an estimate of the condition number of to $p(x)=0$.}

\noindent Assuming $\tilde{x}=1$ as given in 4a.\\
$p(x)=0.0000248$\\
$p(\tilde{x})=-0.0011002$.\\
$\tilde{p}(x)=0.00125$\\.
$\tilde{p}(\tilde{x})=-0.5$.\\
\\
The condition number is $\frac{|(p(\tilde{x})-p(x))/p(x)|}{|(\tilde{x}-x)/x|}$. Plugging in the numbers we get approx. 938.6. This is large and means the problem is ill-conditioned.
\vspace*{0.25in}

\noindent\textbf{Problem 5a: Let $f(x) = x^2 - 8x + 16$. Carry out Newton's using $x_0=12$. Show all steps.}

\noindent $f'(x)=2x-8$\\
$x_1=x_0-\frac{f(x_0)}{f'(x_0)}=12-\frac{64}{16}=8$.\\
$x_2=x_1-\frac{f(x_1)}{f'(x_1)}=8-\frac{16}{8}=6$.\\
$x_3=x_2-\frac{f(x_2)}{f'(x_2)}=6-\frac{4}{4}=5$.\\

\vspace*{0.25in}
\noindent\textbf{Problem 5b: Would you expect the convergence to be linear or quadratic? Explain}

\noindent According to the text, multiple roots converge linearly according to $1-\frac{1}{m}$ with $m$ being the multiplicity.

\end{document}










